\documentclass{article}
\usepackage{preamb/iris}
\usepackage{preamb/heaplang}
\usepackage{graphicx} % Required for inserting images
\usepackage{listings}
\usepackage{ebproof}
\usepackage{parskip}
\usepackage[margin=1.2in]{geometry}

\title{Program Logics Hand-in 4}
\author{Zijun Yu 202203581}
\date{Octobor 2023}

\begin{document}

\renewcommand{\implies}{\Rightarrow}
\newcommand{\kleeneeq}{\ensuremath{\simeq}}
\newcommand{\partfun}{\ensuremath{\rightharpoonup}}
\newcommand{\finparmap}{\overset{\text{fin}}{\rightharpoonup}}
\newcommand{\id}[1]{\ensuremath{\text{id}_{#1}}}
\newcommand{\NN}{\ensuremath{\mathbb{N}}}
\newcommand{\QQ}{\ensuremath{\mathbb{Q}}}
\newcommand{\RR}{\ensuremath{\mathbb{R}}}
\newcommand{\HH}{\ensuremath{\mathbb{H}}}
% \newcommand{\seq}[1]{\ensuremath{\left\{#1_n\right\}_{n=0}^{\infty}}}
\newcommand{\seqn}[1]{\ensuremath{\left(#1\right)_{n=0}^{\infty}}}
\newcommand{\seqns}[1]{\ensuremath{\left\{#1\right\}_{n=0}^{\infty}}}
\newcommand{\sequen}[3]{\ensuremath{{#1 \isetsep #2 \vdash #3}}}
\newcommand{\hastype}[3]{\ensuremath{{#1 \vdash #2 : #3}}}
\newcommand{\closure}[1]{\ensuremath{\text{Cl}\left(#1\right)}}

\newcommand{\timelessjudg}[1]{\proves #1\ \textlog{timeless}}

% semantics
\newcommand{\den}[1]{\ensuremath{\left\llbracket #1 \right\rrbracket}}
\newcommand{\nequal}[1]{\ensuremath{\overset{#1}{=}}}
\newcommand{\cut}[2]{\ensuremath{\left\lfloor #2 \right\rfloor_{#1}}}

\renewcommand{\phi}{\varphi}
\renewcommand{\theta}{\vartheta}
\newcommand{\eps}{\varepsilon}
\renewcommand{\epsilon}{\varepsilon}

\renewcommand{\hom}[3]{\ensuremath{\text{Hom}_{#1}\left(#2,#3\right)}}
\newcommand{\iso}{\cong}
\newcommand{\riso}{\overset{\approx}{\rightarrow}}

\newenvironment{diagram}{\begin{tikzcd}[row sep=1.5cm,column sep=1.5cm]}{\end{tikzcd}}
\newenvironment{largediagram}{\begin{tikzcd}[row sep=2.6cm,column sep=2.6cm]}{\end{tikzcd}}
\newenvironment{smalldiagram}{\begin{tikzcd}[row sep=1cm,column sep=1cm]}{\end{tikzcd}}

% various categories
% \newcommand{\op}[1]{\ensuremath{#1^{\text{op}}}}
\newcommand{\CAT}{\ensuremath{\mathbf{Cat}}}
\newcommand{\sets}{\ensuremath{\mathbf{Set}}}
\newcommand{\heyt}{\ensuremath{\mathbf{Heyt}}}
% \newcommand{\poset}{\ensuremath{\mathbf{Poset}}}
\newcommand{\PSh}[1]{\ensuremath{\text{PSh}\left(#1\right)}}
\newcommand{\subobj}[1]{\ensuremath{\mathbf{Sub}\left(#1\right)}}
\newcommand{\subobjf}[2]{\ensuremath{\mathbf{Sub}_{#1}\left(#2\right)}}

\newcommand{\BB}{\ensuremath{\mathbb{B}}}
\newcommand{\Cc}{\ensuremath{\mathbf{C}}}
\newcommand{\El}{\ensuremath{\mathcal{E}}}
\newcommand{\Sl}{\ensuremath{\mathcal{S}}}
\newcommand{\Ul}{\ensuremath{\mathcal{U}}}
\newcommand{\Dl}{\ensuremath{\mathcal{D}}}
\newcommand{\Fl}{\ensuremath{\mathcal{F}}}
\newcommand{\Pl}{\ensuremath{\mathcal{P}}}
\newcommand{\Tl}{\ensuremath{\mathcal{T}}}
\newcommand{\CC}{\ensuremath{\mathbb{C}}}
\newcommand{\KK}{\ensuremath{\mathbb{K}}}
\newcommand{\PP}{\ensuremath{\mathbb{P}}}
\newcommand{\VV}{\ensuremath{\mathbb{V}}}
\newcommand{\UU}{\ensuremath{\mathbb{U}}}
\newcommand{\DD}{\ensuremath{\mathbb{D}}}
\newcommand{\Ml}{\ensuremath{\mathcal{M}}}
\newcommand{\Vl}{\ensuremath{\mathcal{V}}}
\newcommand{\Il}{\ensuremath{\mathcal{I}}}
\newcommand{\Cl}{\ensuremath{\mathcal{C}}}
\newcommand{\Bl}{\ensuremath{\mathcal{B}}}
\newcommand{\Al}{\ensuremath{\mathcal{A}}}
\newcommand{\Gl}{\ensuremath{\mathcal{G}}}
\newcommand{\Nl}{\ensuremath{\mathcal{N}}}
\newcommand{\AAA}{\ensuremath{\mathbb{A}}}
\newcommand{\EE}{\ensuremath{\mathbb{E}}}
% \newcommand{\TT}{\ensuremath{\mathbb{T}}}

% various combinations of arrows

% separators
\newcommand{\isetsep}{\;\ifnum\currentgrouptype=16 \middle\fi|\;}

% other notations
\newcommand{\powerset}[1]{\ensuremath{\mathcal{P}\left(#1\right)}}
\newcommand{\upred}[1]{\ensuremath{\mathbf{UPred}\left(#1\right)}}
\newcommand{\powup}[1]{\ensuremath{\mathcal{P}^{\uparrow}\left(#1\right)}}
\newcommand{\powdown}[1]{\ensuremath{\mathcal{P}^{\downarrow}\left(#1\right)}}
% \newcommand{\comp}{\circ}
\newcommand{\restr}[2]{\ensuremath{\mathbf{r}_{#2}^{#1}}}
\newcommand{\reindex}[1]{\ensuremath{#1^*}}
\newcommand{\limit}{\varprojlim}
\newcommand{\colimit}{\varinjlim}
\newcommand{\blackbox}{\blacksquare}
\newcommand{\blackdiamond}{\blacklozenge}

% text macros
\newcommand{\ie}{\emph{i.e.,}}
\newcommand{\eg}{\emph{e.g.,}}
\newcommand{\etc}{\emph{etc.}}


% programming language
\newcommand{\proglang}{$\lambda_{\mathrm{ref},\mathrm{conc}}$}
% basic steps of the operational semantics
\newcommand{\stepstopure}{\overset{\mathrm{pure}}{\rightsquigarrow}}
\newcommand{\stepsto}{\rightsquigarrow}
% One-step reduction of thread pools (configurations)
%\newcommand{\cstepsto}{\underset{c}{\stepsto}}
\newcommand{\cstepsto}{\rightarrow}

\newcommand{\Heap}{\textdom{Heap}}
\newcommand{\ECtx}{\textdom{ECtx}}
\newcommand{\TPool}{\textdom{TPool}}
\newcommand{\Config}{\textdom{Config}}

\newcommand{\inl}{\operatorname{inl}}
\newcommand{\inr}{\operatorname{inr}}
\newcommand{\case}[5]{\operatorname{case}(#1,#2.#3,#4.#5)}

% parallel composition operator
\newcommand{\parcomp}{\ensuremath{\mathbin{||}}}

\newcommand{\freevars}[1]{\ensuremath{\text{FV}\left(#1\right)}}

\newcommand{\Iris}{Iris}
\newcommand{\Coq}{Coq}

%%% Local Variables:
%%% mode: latex
%%% TeX-master: "main"
%%% preview-scale-function: 1.2
%%% End:


\maketitle

\section*{Exercise 1}

We introduce $m$ and apply $\textsc{Ht-inv-open}$ and $\text{E-}\later$,
since $l \leftarrow (m+1)$ is an atomic expression (assuming that $m+1$ is a value), SFTS
\begin{equation}
    \boxed{I \ n}^\iota \vdash
    \{\exists m'. \later ( m' \geq n \wedge l \gmapsto m' ) \star ( m \geq n ) \}
    l \leftarrow (m+1)
    \{\_.\later ( \exists m. m \geq n \wedge l \gmapsto m ) \star \text{True}\}_{\mathcal{E}\setminus\iota}
\end{equation}
We then use $\textsc{Ht-exist}$ and introduce $m'$, SFTS
\begin{equation}
    \boxed{I \ n}^\iota \vdash
    \{\later ( m' \geq n \wedge l \gmapsto m' ) \star ( m \geq n ) \}
    l \leftarrow (m+1)
    \{\_.\later ( \exists m. m \geq n \wedge l \gmapsto m ) \star \text{True}\}_{\mathcal{E}\setminus\iota}
\end{equation}
Using $\textsc{Ht-frame-atmoic}$ with $\textsc{Ht-store}$ and other structure rules we have
\begin{equation}
    \boxed{I \ n}^\iota \vdash
    \{\later ( l \gmapsto m' \wedge m \geq n )\}
    l \leftarrow (m+1)
    \{v. v =() \wedge l \gmapsto (m+1) \wedge m \geq n \}_{\mathcal{E}\setminus\iota}
\end{equation}
Because the precondition in (2) implies the precondition in (3), we apply $\textsc{Ht-csq}$,
and we are left to show
$$ v =() \wedge l \gmapsto (m+1) \wedge m \geq n
    \vdash \later ( \exists m. m \geq n \wedge l \gmapsto m ) \star \text{True}$$
This holds as we can take $\exists m$ to be $m+1$ and apply $\text{E-}\later$ and
$\textsc{Later-weak}$.

\section*{Exercise 2}

\begin{prooftree}
    \hypo{P_2 \proves \pvs Q}
    \infer1{P_1 \star P_2 \proves P_1 \star \pvs Q \proves \pvs (P_1 \star Q)}
    \hypo{P_1 \star Q \proves \pvs R}
    \infer1{\pvs (P_1 \star Q) \proves \pvs \pvs R \proves \pvs R}
    \infer2{P_1 \star P_2 \proves \pvs R}
\end{prooftree}

\section*{Exercise 3}

\textbf{HT-TOKEN-ALLOC}

By $\textsc{Ht-csq-vs}$, it suffices to show $S \proves P \Rrightarrow \exists \gamma. \dashbox{T}^\gamma \star P$,
which is equivalent to
$$S \proves \square (P \Rightarrow \pvs (\exists \gamma. \dashbox{T}^\gamma \star P))$$
Because $S \proves True \proves \square True$, by $\textsc{Trans}$, it suffices to show
$$\square True \proves \square (P \Rightarrow \pvs (\exists \gamma. \dashbox{T}^\gamma \star P))$$
Then by $\textsc{Persistently-intro}$, it suffices to show
$$\square True \proves P \Rightarrow \pvs (\exists \gamma. \dashbox{T}^\gamma \star P)$$
which simplifies to
$$P \proves \pvs (\exists \gamma. \dashbox{T}^\gamma \star P)$$
which can be derived from $\textsc{Ghost-alloc}$, $\star$I, and $\textsc{Upd-frame}$.

\textbf{HT-TOKEN-UPDATE-PRE}

By $\textsc{Ht-csq-vs}$, it suffices to show
$$S \proves \dashbox{S}^\gamma \star P \Rrightarrow \dashbox{F}^\gamma \star P$$
By applying the same trick as in $\textsc{Ht-token-alloc}$, it suffices to show
$$\dashbox{S}^\gamma \star P \proves \pvs \left(\dashbox{F}^\gamma \star P\right)$$
which follows from $\textsc{Ghost-update}$, $\star$I, and $\textsc{Upd-frame}$.

\textbf{HT-TOKEN-UPDATE-POST}

By $\textsc{Ht-csq-vs}$, what we are left to show is exactly the same as in $\textsc{Ht-token-update-pre}$.

\section*{Exercise 4}
By $\textsc{Ht-load}$, we know that
$$
    \left\{\later l \gmapsto n  \right\}
    !l
    \left\{v.v=n \star l \gmapsto n \right\}
$$
By $\textsc{Ht-csq}$, we further have
$$
    \left\{\later l \gmapsto n  \right\}
    !l
    \left\{v.\left(v=n \vee v=n+1 \star \dashbox{F}^{\gamma_1}\right) \star l \gmapsto n \right\}
$$
Then by $\textsc{Ht-frame-atomic}$ and $\star$I, we know that
$$
    \left\{\later \left( l \gmapsto n \star \dashbox{S}^{\gamma_1} \right) \star \dashbox{$\frac{1}{2}$}^{\gamma_2} \right\}
    !l
    \left\{v.\left(v=n \vee v=n+1 \star \dashbox{F}^{\gamma_1}\right) \star l\gmapsto n \star \dashbox{S}^{\gamma_1} \star \dashbox{$\frac{1}{2}$}^{\gamma_2}\right\}
$$
Because $l\gmapsto n \star \dashbox{S}^{\gamma_1} \proves \later \left(l \mapsto n \star \dashbox{S}^{\gamma_1} \right) \proves \later I(\gamma_1, \gamma_2, n)$,
by $\textsc{Ht-csq}$, we have that
$$
    \left\{\later \left( l \gmapsto n \star \dashbox{S}^{\gamma_1} \right) \star \dashbox{$\frac{1}{2}$}^{\gamma_2} \right\}
    !l
    \left\{v.\left(v=n \vee v=n+1 \star \dashbox{F}^{\gamma_1}\right) \star \later I(\gamma_1, \gamma_2, n) \star \dashbox{$\frac{1}{2}$}^{\gamma_2}\right\}
$$
which is what we need to show.

\end{document}
