\documentclass{article}
\usepackage{preamb/iris}
\usepackage{preamb/heaplang}
\usepackage{graphicx} % Required for inserting images
\usepackage{listings}
\usepackage{ebproof}
\usepackage{parskip}
\usepackage[margin=1.2in]{geometry}

\title{Program Logics Hand-in 2}
\author{Zijun Yu 202203581}
\date{September 2023}

\begin{document}
\include{preamb/macros.tex}

\maketitle

\section*{Exercise 1}

\begin{center}
    \begin{prooftree}
        \hypo{S \vdash \{l \gmapsto 1 \} l \leftarrow 2 \{v. \wedge l \gmapsto 2 \}}
        \infer1[\textsc{Ht-frame-invalid}]{S \vdash \{l \gmapsto 1 \wedge l \gmapsto 1 \} l \leftarrow 2 \{v. l \gmapsto 2 \wedge l \gmapsto 1 \}}
    \end{prooftree}
\end{center}

We got to a point where we have $l \gmapsto 1 \wedge l \gmapsto 2$, suggesting that the rule is unsound.

\section*{Exercise 2}

We first prove the following \textsc{Ht-pre-eq} rule:

\begin{center}
    \begin{prooftree}
        \infer0{\Gamma, x \vdash Q : Prop}
        \hypo{\Gamma \mid S \vdash \{P[v/x]\} e[v/x] \{u.Q[v/x]\}}
        \infer1{\Gamma, x \mid S \wedge x=v \vdash \{P[v/x]\} e[v/x] \{u.Q[v/x]\}}
        \infer0{\Gamma, x \mid S \wedge x=v \vdash x=v}
        \infer3[Eq]{\Gamma, x \mid S \wedge x=v \vdash \{P\} e \{u.Q\}}
        \infer1[\textsc{Ht-eq}]{\Gamma, x \mid S \vdash \{x=v \wedge P\} e \{u.Q\}}
    \end{prooftree}
\end{center}

Then we prove \textsc{Ht-bind-det}:

\begin{center}
    \begin{prooftree}
        \hypo{S \vdash \{P\} e \{x.x=u \wedge Q\} }
        \hypo{S \vdash \{Q[u/x]\} E[u] \{w.R\}}
        \infer1[\textsc{Ht-pre-eq}]{x \mid  S \vdash \{x=u \wedge Q\} E[x] \{w.R\}}
        \infer1{S \vdash \forall x. \{x=u \wedge Q\} E[x] \{w.R\}}
        \infer2[\textsc{Ht-bind}]{S \vdash \{P\} E[e] \{w.R\}}
    \end{prooftree}
\end{center}

We cannot use this rule in the case where $e=ref(0)$ as we will
not be able to prove the middle antecedent, i.e.
$S \vdash \{P\} ref(0) \{x.x=u \wedge Q\} $, because the result
of $ref(0)$ is not deterministic and we might have assigned $u$
to a location that is already allocated.

\section*{Exercise 3}

In this definition of linked list, each node is a pair of the value
and either null (injection left) or the pointer (location) to the next node.
If we expand the recursive definition of isList, by the seperating
conjunction, we are ensured that all the locations are different, hence
the list is acyclic.

If we use ordinary conjunction, we can have a cyclic list, for example,
let $l = \Inr (hd)$ where $hd \gmapsto (1, l)$, then $l$ is a cyclic list.
And $l$ is a valid input to the predicate $\text{isList}$, since we can choose
all the $l'$ to be $l$ when expanding the recursion.

\section*{Exercise 4}

\subsection*{Part 1}

We proceed by the $\textsc{Ht-Rec}$ rule and we consider two cases:
we use the fact that the sequence $xs$ is either empty $[]$ or
has a head $x$ followed by another sequence $xs'$.

In the first case we need to show that
$$
    \forall ys, l, l'. \{\text{isList} \ l \ [] \star \text{isList} \ l' \ ys\}
    \langkw{match} \ l \ \langkw{with} ... \{v. \text{isList} \ v \ ([] ++ ys)\}
$$

Because $\text{isList \ l []} \equiv l = \Inl ()$, then by $\textsc{Ht-match}$
and $\textsc{Ht-csq}$, our goal becomes
$$
    \forall ys, l'. \{l = \Inl () \star \text{isList} \ l' \ ys\} l' \{v. \text{istList} \ v \ ys\}
$$

which obviously holds. (To be precise, we need to discuss the two cases of $ys$
again, and in each case, use $\textsc{Ht-pre-eq}$ and $\textsc{Ht-ret}$.)

In the second case, we need to show that
$$
    \forall x, xs', ys, l, l'. \{\text{isList} \ l \ (x :: xs') \star \text{isList} \ l' \ ys\}
    \langkw{match} \ l \ \langkw{with} ... \{v. \text{isList} \ v \ ((x :: xs') ++ ys)\}
$$
where
$$
    \text{isList} \ l \ (x :: xs') \equiv \exists hd, t. l=\Inr hd * hd \gmapsto (x, t) * \text{isList} \ t \ xs'
$$

Thus we can prove
$$
    \{l = \Inr hd \star hd \gmapsto (x, t) \star \text{isList} \ t \ xs' \} l \{v. v= \Inr hd \star l = \Inr hd \star hd \gmapsto (x, t) \star \text{isList} \ t \ xs' \}
$$

In this case, the second branch of the match is taken, and by the derived rule from
exercise 4.10, we proceed to verify the body of the second branch.

We initially have

$\{l = \Inr hd \star hd \gmapsto (x, t) \star \text{isList} \ t \ xs' \}$.

After processing $\Let{p}={!hd}in$, we have

$\{p=(x,t) \star l = \Inr hd \star hd \gmapsto (x, t) \star \text{isList} \ t \ xs' \}$

After processing $\Let{r}={\langkw{append}}in$, using the induction hypothesis introduced
by $\textsc{Ht-Rec}$, we have

$\{\text{isList} \ r \ (xs' ++ ys) \star p=(x,t) \star l = \Inr hd \star hd \gmapsto (x, t) \star \text{isList} \ t \ xs' \}$

After processing $hd \leftarrow (\pi_1 p, r)$, we have

$\{\text{isList} \ r \ (xs' ++ ys) \star p=(x,t) \star l = \Inr hd \star hd \gmapsto (x, r) \star \text{isList} \ t \ xs' \}$

and we are left to prove that the value of the $\Inr hd$ is the sequence we want, i.e.
$\text{isList} \ v \ (xs ++ ys)$, which is true since $x::(xs'++ys) = (x::xs')++ys = xs++ys$.


\subsection*{Part 2}

No. Because if we use this specification on $\langkw{append} \ l \ l$, we will get a
cyclic list as the result, and we will not be able to prove
$\text{isList} \ v \ (xs ++ xs)$.

\end{document}